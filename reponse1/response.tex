\documentclass{article}
	\def\papertitle{If you build it who will come? Equity analysis of park system changes during COVID-19 using passive origin-destination data}
	%\def\authors{Gregory S. Macfarlane, Christian Hunter, Austin Martinez, and Elizabeth Smith}
	\def\journal{Journal of Transport and Land Use}
	\def\paperid{8606}
% Define title defaults if not defined by user
\providecommand{\lettertitle}{Author Response to Reviews of}
\providecommand{\papertitle}{Title}
\providecommand{\authors}{Authors}
\providecommand{\journal}{Journal}
\providecommand{\paperid}{--}

\usepackage[includeheadfoot,top=20mm, bottom=20mm, footskip=2.5cm]{geometry}
\usepackage{booktabs}

% Typography
\usepackage[T1]{fontenc}
\usepackage{times}
%\usepackage{mathptmx} % math also in times font
\usepackage{amssymb,amsmath}
\usepackage{microtype}
\usepackage[utf8]{inputenc}
% Misc
\usepackage{graphicx}
\usepackage[hidelinks]{hyperref} %textopdfstring from pandoc
\usepackage{soul} % Highlight using \hl{}

\usepackage[round,authoryear]{natbib}

% Table

\usepackage{adjustbox} % center large tables across textwidth by surrounding tabular with \begin{adjustbox}{center}
\renewcommand{\arraystretch}{1.5} % enlarge spacing between rows
\usepackage{caption}
\captionsetup[table]{skip=10pt} % enlarge spacing between caption and table

% Section styles

\usepackage{titlesec}
\titleformat{\section}{\normalfont\large}{\makebox[0pt][r]{\bf \thesection.\hspace{4mm}}}{0em}{\bfseries}
\titleformat{\subsection}{\normalfont}{\makebox[0pt][r]{\bf \thesubsection.\hspace{4mm}}}{0em}{\bfseries}
\titlespacing{\subsection}{0em}{1em}{-0.3em} % left before after

% Paragraph styles

\setlength{\parskip}{0.6\baselineskip}%
\setlength{\parindent}{0pt}%

% Quotation styles

\usepackage{framed}
\let\oldquote=\quote
\let\endoldquote=\endquote
\renewenvironment{quote}{\begin{fquote}\advance\leftmargini -2.4em\begin{oldquote}}{\end{oldquote}\end{fquote}}

\usepackage{xcolor}
\newenvironment{fquote}
  {\def\FrameCommand{
	\fboxsep=0.6em % box to text padding
	\fcolorbox{black}{white}}%
	% the "2" can be changed to make the box smaller
    \MakeFramed {\advance\hsize-2\width \FrameRestore}
    \begin{minipage}{\linewidth}
  }
  {\end{minipage}\endMakeFramed}

% Table styles

\let\oldtabular=\tabular
\let\endoldtabular=\endtabular
\renewenvironment{tabular}[1]{\begin{adjustbox}{center}\begin{oldtabular}{#1}}{\end{oldtabular}\end{adjustbox}}


% Shortcuts

%% Let textbf be both, bold and italic
%\DeclareTextFontCommand{\textbf}{\bfseries\em}

%% Add RC and AR to the left of a paragraph
%\def\RC{\makebox[0pt][r]{\bf RC:\hspace{4mm}}}
%\def\AR{\makebox[0pt][r]{AR:\hspace{4mm}}}

%% Define that \RC and \AR should start and format the whole paragraph
\usepackage{suffix}
\long\def\RC#1\par{\makebox[0pt][r]{\bf RC:\hspace{4mm}}#1\par} %\RC
\WithSuffix\long\def\RC*#1\par{\textbf{\textit{#1}}\par} %\RC*
\long\def\AR#1\par{\makebox[0pt][r]{AR:\hspace{10pt}}\textit{#1}\par} %\AR
\WithSuffix\long\def\AR*#1\par{\textit{#1}\par} %\AR*


%%%
%DIF PREAMBLE EXTENSION ADDED BY LATEXDIFF
%DIF UNDERLINE PREAMBLE %DIF PREAMBLE
\RequirePackage[normalem]{ulem} %DIF PREAMBLE
\RequirePackage{color}\definecolor{RED}{rgb}{1,0,0}\definecolor{BLUE}{rgb}{0,0,1} %DIF PREAMBLE
\providecommand{\DIFadd}[1]{{\protect\color{blue}\uwave{#1}}} %DIF PREAMBLE
\providecommand{\DIFdel}[1]{{\protect\color{red}\sout{#1}}}                      %DIF PREAMBLE
%DIF SAFE PREAMBLE %DIF PREAMBLE
\providecommand{\DIFaddbegin}{} %DIF PREAMBLE
\providecommand{\DIFaddend}{} %DIF PREAMBLE
\providecommand{\DIFdelbegin}{} %DIF PREAMBLE
\providecommand{\DIFdelend}{} %DIF PREAMBLE
%DIF FLOATSAFE PREAMBLE %DIF PREAMBLE
\providecommand{\DIFaddFL}[1]{\DIFadd{#1}} %DIF PREAMBLE
\providecommand{\DIFdelFL}[1]{\DIFdel{#1}} %DIF PREAMBLE
\providecommand{\DIFaddbeginFL}{} %DIF PREAMBLE
\providecommand{\DIFaddendFL}{} %DIF PREAMBLE
\providecommand{\DIFdelbeginFL}{} %DIF PREAMBLE
\providecommand{\DIFdelendFL}{} %DIF PREAMBLE
%DIF END PREAMBLE EXTENSION ADDED BY LATEXDIFF

\begin{document}

% Make title
{\Large\bf \lettertitle}\\[1em]
{\huge \papertitle}\\[1em]
{\authors}\\
\textit{\journal}, \texttt{\paperid}\\
\hrule

% Legend
\hfill {\bfseries RC:} Reviewer Comment,\(\quad\) AR: \emph{Author Response}, \(\quad\square\) Manuscript text


We are grateful to the two anonymous reviewers for their review and consideration
of the manuscript.  In this document we have highlighted \DIFadd{additions to the text of the manuscript with blue letters} and
\DIFdel{text removed from the manuscript with red letters.}

As a general change from the first manuscript, we have updated the library used to
generate our maps, resulting in minor changes to the presentation but not the
content. We also pulled new data from OpenStreetMap for the park amenities. This
resulted in small changes to the estimated model coefficients but none are
substantial enough to  affect the findings. A few minor typographical errors
have also been corrected.

\section{Reviewer B}

\RC This study aims to evaluate how converting roadway facilities into open pedestrian spaces affects park accessibility and the socio-spatial distribution benefits of those changes. The authors use StreetLight data from Alameda County, California to estimate a park activity location choice model, from which they construct their park accessibility measure. They find that Alameda County’s street conversions have disproportionately benefited Black, Hispanic, and low-income households. Overall, this is a good paper with a path to publication.

\AR We are grateful that the reviewer found our paper worthwhile.

\subsection{Literature}

\RC The intro paragraph to the Literature section only mentions two literatures when in fact there are three subsections of literature discussed. It would be useful to include a synthesis of the literature on sociodemographic variation in park utility at the end of section 2.1. Good review of the literature on measuring accessibility.

\AR Originally, the "two literatures" referred to the parks preference
literature  and the accessibility literature. We agree that the  paragraph is
unnecessarily confusing. It now reads:

\begin{quote}
	Understanding the equity benefit distribution of park access requires us to
consider \DIFdelbegin \DIFdel{two integrated but rather distinct }\DIFdelend \DIFaddbegin \DIFadd{multiple }\DIFaddend literatures. First, we consider the
disparity in park utility perception among different populations. We
subsequently consider quantitative techniques to evaluate the access that
individuals have to park facilities. \DIFaddbegin \DIFadd{Finally, we consider recent research
documenting and analyzing street conversions instigated by the COVID-19
pandemic.
}\DIFaddend

\end{quote}

\subsection{Methodology}

\RC Utility-based accessibility is a good framework for analyzing the distributional benefits of street conversions, and the authors generally do a good job of explaining and supporting that methodological choice. However, the paper would benefit from additional explanation on at least three issues.

\RC First, it is unclear to me whether the same 10 additional parks are used for each synthetic choice maker’s choice set, or whether a new random selection of 10 additional parks is done for each choice maker.

\AR It is a different set of 10 parks for each synthetic choice maker. We have clarified this in the manuscript as

\begin{quote}
	For this reason we randomly sample 10 additional parks
	to serve as the non-chosen alternatives\DIFdelbegin \DIFdel{for our }\DIFdelend \DIFaddbegin \DIFadd{, with a different set of 10 parks for
	each }\DIFaddend synthetic choice maker.
\end{quote}

\RC Second, the authors should discuss how reasonable their selection of choice sets is. I understand that it’s impractical to estimate a choice model with 500 alternatives. But how realistic are the choice sets of 10 randomly selected parks? Is there a way to generate more realistic choice sets without biasing the model results?

\AR The estimates of a choice model with random sampling of alternatives are
known to be unbiased but inefficient relative to either  a complete choice set
or a more carefully constructed sampling scheme \citep{train2009}. In this case,
we observe that it would be difficult to create a realistically weighted
alternative set (the ten  nearest parks? The three nearest parks, two major
regional parks, and five at random? Something else?) that is itself unbiased. We
also observe that the problem of efficiency is not a challenge in this analysis;
virtually all estimated parameters would not change their level of significance
or interpretation with marginally more precise estimates. We have added the following
to the methodology discussion:

\begin{quote}
	Such
 random sampling of alternatives reduces the efficiency of the estimated
 coefficients but the coefficients remain unbiased \citep{train2009}\DIFaddbegin \DIFadd{; a more elegant
 sampling approach might have resulted in smaller estimated standard errors,
 but the estimation results (presented below) suggest this is not a concern in
 this application}\DIFaddend .
\end{quote}

\RC Third, the authors should provide more justification for their choice of cost coefficient since there are no cost variables in their model. Why is the chosen cost coefficient reasonable and not arbitrary? Did the authors explore other options?

\AR Because we are only using the cost coefficient to scale the logsum utility
benefits and not to do rigorous cost-benefit analysis, any reasonable scalar would be sufficient.
That said, we suggest that using the cost coefficient present in the region's
activity-based travel model for relevant trips could hardly be characterized as
arbitrary. Nevertheless, we have re-written the relevant paragraph to be more clear. In
doing so, we have adopted the MTC model documentation directly, which differs
modestly from the ActivitySim open-source model derived from it. This results in
a new parameter value, and consequently scaled changes to the user benefits estimates.
Though the numerical benefits have changed, our central finding related to benefit
distribution has not been affected. The methodology to review

\begin{quote}
	As a substitution, we use an estimate of the cost coefficient obtained from the
	\DIFdelbegin \DIFdel{open-source activity-based travel demand model ActivitySim \mbox{%DIFAUXCMD
	\citep{activitysim}}\hspace{0pt}%DIFAUXCMD
	, which is itself
	based on the regional travel model employed by the }\DIFdelend Metropolitan Transportation Commission (MTC\DIFdelbegin \DIFdel{), the }\DIFdelend \DIFaddbegin \DIFadd{, }\DIFaddend San Francisco Bay regional MPO\DIFdelbegin \DIFdel{. ActivitySim uses a cost
	coefficient of \(-0.6\) divided by the each simulated agent's value of time to
	determine destination choices for non-work trips .}\footnote{\DIFdel{To be precise, this is the
	  cost coefficient on the mode choice model for social, recreational, and other
	  trip purposes, which influences destination choice through a logsum-based
	  impedance term.}} %DIFAUXCMD
	\addtocounter{footnote}{-1}%DIFAUXCMD
	\DIFdel{In ActivitySim, as in most }\DIFdelend \DIFaddbegin \DIFadd{)
	``Travel Model One'' }\DIFaddend activity-based \DIFdelbegin \DIFdel{travel models, the value of time is considered to vary with an }\DIFdelend \DIFaddbegin \DIFadd{travel demand model \mbox{%DIFAUXCMD
	\citep{mtctm1}}\hspace{0pt}%DIFAUXCMD
	. In this model,
	the utility of destination choice for social and recreational trips uses
	the mode choice model logsum as an impedance measure.
	The mode choice model cost coefficient varies with each }\DIFaddend individual's \DIFdelbegin \DIFdel{income, but in this
	aggregate destination choice model, an aggregate }\DIFdelend value of time\DIFdelbegin \DIFdel{will suffice}\DIFdelend .
	The average value of time in the synthetic population for the \DIFdelbegin \DIFdel{Bay Area }\DIFdelend \DIFaddbegin \DIFadd{calibration
	scenario }\DIFaddend is \$7.75 per hour\DIFdelbegin \DIFdel{, resulting in a cost coefficient on the destination choice utility of
	\(-0.215\). Dividing the difference in accessibility logsums by the negative of this value gives an initial estimate of
	the monetary value of the policy to each
	park user}\DIFdelend . \DIFaddbegin \DIFadd{Dividing that value of time by the in-vehicle
	travel time parameter of \(-0.018\) results in an implied mode choice cost coefficient of
	\((-0.018 /min * 60 min / hr)/ (7.75 \$/hr) = -0.139 / \$\).
	}\DIFaddend
\end{quote}

\subsection{Results}

\RC Should the ratio of coefficients be reversed (p. 16)? Shouldn't the coefficient for the desired amenity (in this case increased park acreage) be the numerator, divided by the coefficient for the cost (in this case distance)? Using the ratios reported in the manuscript would imply that park visitors living in block groups with a high proportion of Black and low-income residents would be willing to travel farther for bigger parks than others, contrary to the authors’ conclusion that they are “considerably more sensitive to the distance to a park” (p. 20).

\RC In the equity analysis, the authors should explain how they calculated the benefit ``for simply having more options'' (p. 20). They should also provide a citation for the proposition that ``[o]ne property of logsum-based accessibility terms is that there is some benefit for simply having more options'' (p. 20).

\AR We are unaware of a specific citation for this claim, but the assertion is trivially derived from the logsum equation.
If one new alternative $q$ is added to the choice set, then the logsum becomes
 $\ln\left(\sum_{j \in J}e^{u_j} + e^{u_q}\right)$; because $e^x > 0$ everywhere,
and because $\ln(x)$ is monotonically increasing, then it is always true that
$\ln\sum_{j \in J}e^{u_j} < \ln\left(\sum_{j \in J}e^{u_j} + e^{u_q}\right)$. We
have added the following sentence to the mathematical description,

\begin{quote}
	\DIFaddbegin \DIFadd{Note also that the logsum increases
 with the size of the choice set: if a new alternative \(q\) is added to \(J\), then
 \(\ln\sum_{j\in J}\exp(V_j) < \ln(\sum_{j\in J}\exp(V_j) + \exp(V_q))\) for
 any value of \(V_q\).
 }\DIFaddend

\end{quote}

\RC I suggest caveating the presentation of monetary benefits. Many readers might find these numbers difficult to understand and/or arbitrary (despite the preceding explanation of calculating consumer surplus in the methods section). How real are these dollar values? Do they suggest that the total value of the street closures is just \$664,628? From my perspective, the strength of the consumer surplus analysis is not in presenting dollar values. The strength is that it allows an estimation of the distribution of benefits, which could be done regardless of the value of the cost coefficient.

\AR This is a very good point. One advantage of converting a consumer surplus into
monetary terms is that it transforms what is a fairly esoteric quantity --- the
utility benefits derived via a choice model logsum --- into a value that policy
maker and the public can understand. Of course, our own edits to the manuscript
show how heavily this value depends on assumed or estimated cost coefficients
which have some uncertainty distribution associated with them. We have added the
following caveat to the limitations section:

\begin{quote}
	\DIFaddbegin \DIFadd{The monetary benefits we present in this analysis are heavily dependent on
	two separate assumptions. First, reasonable researchers might have selected
	different values of time or cost coefficients. Second, the decision to assign
	one benefit to each household could also have been made
	differently. A change in either assumption would lead to a highly different
	total benefits estimate, but it would not change the distribution of the
	benefits, which is the objective of this study. At some level, converting the
	esoteric measure of choice model logsums into a unit that can be conveniently
	compared against other policies is desirable to help the public and policy
	makers evaluate such decisions. Further research should establish guidelines and
	practices for applying accessibility logsums in monetary cost-benefit analyses.
	}

	\DIFaddend

\end{quote}

\RC The block group segmentations indicate that the utility equations could be different for different groups. That also affects the equity analysis. It is likely that not all residents of each block group will benefit similarly from the COVID-19 street closures. This would affect the distributional equities. I suggest mentioning this caveat.


\section{Reviewer E}


\RC Thank you for the opportunity to review this manuscript. The authors assess the policy of converting roads into pedestrian open spaces streets in Alameda County, California, from an equity perspective. One of the main contributions from the manuscript is the clever use of travel behaviour data collected through mobile phone sensors to answer questions about equity at the intersection of mobility and land use planning. Another interesting contribution stems from their use of utility-based accessibility measures to address a timely theme that cuts across transport policy and questions of distributive justice. Despite having access to a wealth of data and proposing to answer an interesting research question with potential contributions to policymaking, the paper still needs substantial work to be considered for publication in the Journal of Transport and Land Use. I expect the comments I provide below help improve the manuscript.

\AR We appreciate that the reviewer found our methodology ``clever'' and our contribution
``interesting.'' We are similarly grateful for the constructive criticism the
reviewer provided.


\subsection{Surrounding Land Uses}

\RC I am not persuaded by either the methodological choices made by the authors or their engagement with the academic literature on accessibility. Perhaps the most critical potential threat to internal validity consists of the author's assumption that residents in Alameda County will visit the recently pedestrianized spaces motivated by the same factors that encourage them to visit parks: recreation, exercise, and social interaction. One possible avenue to overcome this limitation could be to account for land uses, for instance, restaurants or retail spaces, around the destinations included in the analysis as a proxy for unobserved differences in trip purposes (e.g., shopping, exercising, and leisure) unavailable in passively collected mobile phones data. These factors, which can be paired with amenities found in parks, were not accounted for in the models presented, neither the potential consequences for their omissions in the conclusions inferred from the model results.

\subsection{Travel Mode of Access}

\RC Another strong assumption that also poses a tread to the research's internal validity is that residents who visit open spaces, including the ones the authors focus on, is that they will travel to these places by foot, and therefore penalizing all individuals by distance instead of travel time. Of course, accounting for travel times requires access to mode choice, which some travel behaviour and physical activity mobile applications can detect. One alternative to mitigate this problem, assuming that there is no information about mode choice, is to subset from the dataset those who had a higher likelihood of had conducted their trip by foot. If this was neither an alternative, it is imperative to include a paragraph discussing how the assumption that all walked to their destinations may compromise the analysis and conclusions.

\subsection{Causal Effects}

\RC Although the motivation is a particular policy and the word 'change' is in the title, it caught my attention that the authors' research design overlooks the time dimension. While cross-sectional analyses like the one presented can contribute to the academic literature, the paper does not account for time. For instance, classifying trips by whether they occurred before and after streets were banned to cars and open to only pedestrians and cyclists will provide results more attuned to the paper's motivation. Since the research design employed prevents authors from linking the policy in question with particular behaviours and preferences, any language suggesting cause-and-effect must be edited to reflect the nature of the research design employed.

\subsection{Accessibility Theory}

\RC Regarding the concept of accessibility and the particular ways scholars have attempted to operationalize it, the authors need to explain better why cumulative opportunities measures are inferior in the context of the research question the manuscript attempts to answer. It is not enough to say: "However, they [cumulative opportunity measures] may be too simple, especially concerning trip costs near the threshold." For example, a person living in front of the park may have the same accessibility level that another living 9 minutes away from the same park if the objective of the analysis is to measure how many parks can be accessed within a predefined (or normative) travel time threshold. I recommend expanding this section of the literature review to other works that move from pure methodological discussions to more theoretical ones. One start point could be the paper titled "Measuring accessibility: positive and normative implementations of various accessibility indicators" by \citet{paez2012measuring}. The work of transportation scholars Rafael Pereira or Ahmed El-Geneidy on accessibility may be of help as well.

\subsection{Equity Evaluation}

\RC Another significant omission is the definition of equity that guides the analysis. Please refer to the paper titled "Distributive justice and equity in transportation" by \citet{pereira2017distributive} for a thorough discussion on the issue. Other two relevant sources are the \citet{taylor2009paying} article "Paying for Transportation: What's a Fair Price?" and "Environmental Injustice and Transportation: The Claims and the Evidence" by \citet{schweitzer2004environmental}.

\bibliography{library}
\bibliographystyle{abbrvnat}
\end{document}
