\documentclass[]{elsarticle} %review=doublespace preprint=single 5p=2 column
%%% Begin My package additions %%%%%%%%%%%%%%%%%%%
\usepackage[hyphens]{url}

  \journal{Submitted to Transportation Research Part C: Emerging Technologies} % Sets Journal name


\usepackage{lineno} % add
\providecommand{\tightlist}{%
  \setlength{\itemsep}{0pt}\setlength{\parskip}{0pt}}

\usepackage{graphicx}
\usepackage{booktabs} % book-quality tables
%%%%%%%%%%%%%%%% end my additions to header

\usepackage[T1]{fontenc}
\usepackage{lmodern}
\usepackage{amssymb,amsmath}
\usepackage{ifxetex,ifluatex}
\usepackage{fixltx2e} % provides \textsubscript
% use upquote if available, for straight quotes in verbatim environments
\IfFileExists{upquote.sty}{\usepackage{upquote}}{}
\ifnum 0\ifxetex 1\fi\ifluatex 1\fi=0 % if pdftex
  \usepackage[utf8]{inputenc}
\else % if luatex or xelatex
  \usepackage{fontspec}
  \ifxetex
    \usepackage{xltxtra,xunicode}
  \fi
  \defaultfontfeatures{Mapping=tex-text,Scale=MatchLowercase}
  \newcommand{\euro}{€}
\fi
% use microtype if available
\IfFileExists{microtype.sty}{\usepackage{microtype}}{}
\usepackage{natbib}
\bibliographystyle{plainnat}
\usepackage{longtable}
\ifxetex
  \usepackage[setpagesize=false, % page size defined by xetex
              unicode=false, % unicode breaks when used with xetex
              xetex]{hyperref}
\else
  \usepackage[unicode=true]{hyperref}
\fi
\hypersetup{breaklinks=true,
            bookmarks=true,
            pdfauthor={},
            pdftitle={Developing a Park Activity Location Choice Model from Passive Origin-Destination Data Tables},
            colorlinks=false,
            urlcolor=blue,
            linkcolor=magenta,
            pdfborder={0 0 0}}
\urlstyle{same}  % don't use monospace font for urls

\setcounter{secnumdepth}{5}
% Pandoc toggle for numbering sections (defaults to be off)


% Pandoc header
\usepackage{booktabs}



\begin{document}
\begin{frontmatter}

  \title{Developing a Park Activity Location Choice Model from Passive Origin-Destination Data Tables}
    \author[Brigham Young University]{Gregory Macfarlane\corref{1}}
   \ead{gregmacfarlane@byu.edu} 
    \author[]{Teresa Tapia}
   \ead{teresa.tapia@streetlightdata.com} 
    \author[Harvard University]{Carole Turley-Voulgaris}
   \ead{cat@example.com} 
      \address[Brigham Young University]{Civil and Environmental Engineering Department, 430 Engineering Building, Provo, Utah 84602}
    \address[Harvard University]{Some Other Place}
      \cortext[1]{Corresponding Author}
  
  \begin{abstract}
  Parks provide important benefits to those who live near them, in the form of improved property values, health outcomes, etc.; nevertheless, measuring and understanding who lives near a park is an open research question. In particular, it is not well understood which park individuals will choose to use when given a choice among a set of nearby parks of varying sizes and at varying distances from their home. In this paper we present a park activity location choice model estimated from a passive origin-destination dataset --- supplied by StreetLight Data, Inc.~--- representing trips to parks and green spaces in Alameda County, California. The estimated model parameters reveal heterogeneous preferences for park size and willingness-to-travel across block-group level socioeconomic segmentation: Specifically, high-income block groups appear more positively attracted to larger parks, and block groups with a high proportion of ethnic minority individuals are more likely to select nearby parks. The findings have importance for understanding recreational access among different populations, and the methodology more generally supplies a potential template for using passive data products within travel modeling.
  \end{abstract}
   \begin{keyword} Accessibility Passive Data Location Choice\end{keyword}
 \end{frontmatter}

\hypertarget{intro}{%
\section{Introduction}\label{intro}}

Parks and other green spaces generate immense value for the public who are able
to access them. The \citet{CityParksAlliance} categorizes the observed benefits of
urban parks as encouraging active lifestyles \citep{Bancroft2015}, contributing to
local economies, aiding in stormwater management and flood mitigation,
improving local air quality, increasing community engagement \citep{Madzia2018}, and
enhancing public equity.

Nevertheless, understanding and quantifying these benefits depends in many
cases on identifying who lives near the parks and is therefore able to access
them. Many previous studies \citep[e.g.,][]{Richardson2012} rely on comparison of total
greenspace across metropolitan areas; this methodology may not adequately
control for city-level fixed effects and it may ignore the potentially
inequitable distribution of park space within a region. Studies focusing on
access within metropolitan areas typically assume that people living within a
certain distance or travel time threshold have access to a park, or examine the
quantity of park space within one's own arbitrarily defined ``neighborhood''
\citep[Stark2014]{Mitchell2008}. But these methods do not account for the fact that
some people will travel to other parks to perform recreational activities. A
more holistic measure that continuously measures access across multiple
preference dimensions is desirable.

An appealing solution would be to examine and model the activity location
choices of park users. Such a model would help researchers understand how
individuals of different backgrounds and preferences value different park
amenities. Further, the logsums of a location choice model provide a continuous
measure of accessibility that explicitly accounts for such variation
\citep{DeJong2007}. Unfortunately, park choice models of this form are rare in the
literature. Travel demand models built for infrastructure forecasting are a
common way to generate such accessibility logsums, but these models group many
different kinds of social and recreational trips together \citep{nchrp716}. Further,
the attraction term for such trip purposes is commonly a function of the retail
or service employment or the number of households at the destination; a typical
park or green space has neither employees nor residents. Finally, many regional
household travel surveys are oriented towards an average weekday travel
pattern, and many park trips occur irregularly or on weekends.

In this paper we present a park destination choice model where individuals
living in Alameda County, California choose among parks in the same county. The
individuals are constructed from passive data that was derived from mobile
devices and processed using algorithms developed by StreetLight Data, Inc.~The
origin location points are inferred residence block groups for unique devices
and the destination points are geofenced polygons representing green and open
spaces. The individuals' choice of park location is conditioned on the distance
from the block group to the parks in the choice set as well as the size of each
park; market segmentation allows for heterogeneous responses between ethnic
groups and income strata.

The paper proceeds in the following manner: A discussion of prior attempts to
study park choice and employ passive origin-destination data in the literature
is given directly. The Methodology section presents the data gathering and
cleaning efforts as well as the econometric location choice model. The Results
section presents the estimated model coefficients and a discussion of the
findings, as well as a model validation exercise. After presenting limitations
and associated avenues for future research, a final Conclusions section
outlines the contributions of this study for recreational trip modeling and
location choice modeling more generally.

\hypertarget{literature}{%
\section{Literature}\label{literature}}

Understanding who has access to parks is a long-standing question across
multiple scientific disciplines. Researchers specializing in recreation
management, public health, urban planning, ecology, and civil engineering have
all played a role in shaping our collective understanding of park design,
access, and use. A complete review of all of these fields is not warranted for
the scope of this paper, but some recent findings are worth discussion.

A popular measure of park accessibility is the Trust for Public Land's
``ParkScore'' statistic \citep{parkscore2019}. ParkScore considers the share of the
population that resides within a 10-minute walk of a green space using a
sophisticated network routing algorithm, in combination with the total city
green space, investment, and amenities weighted against the socioeconomic
characteristics of the population outside of the 10-minute walk threshold. The
resulting score is a convenient quantitative tool in estimating the relative
quality of green space access across cities \citep{Rigolon2018}. It may be less
useful at identifying the comparative quality of access within a city,
particularly as more than 95\% of residents in many large metropolitan areas like
San Francisco and New York live within the binary 10-minute walk threshold. The
Centers for Disease Control and Prevention (CDC) has developed an ``Accessibility
to Parks Indicator'' along similar lines \citep{Ussery2016}, calculating the share of
the population living within a half-mile of a park for each county in the U.S.

There is recognition that park access should in some way be linked with park
use. After all, a park that has many visitors must by definition be accessible
to those visitors. \citet{McCormack2010} provide a comprehensive review of this
literature; it is sufficient here to note that most studies find a complicated
interplay between park size, maintenance, facilities, and travel distance. Many
of these attributes are incorporated into ParkIndex \citep{Kaczynski2016}, which
estimates the resident park use potential within \(100 m^2\) grid cells, based on
a household park use survey in Kansas City.

From a transportation engineering perspective, the park use potential measured
by ParkIndex is not dissimilar from a park trip production potential. Along
these lines, the question of park use is a destination choice problem, where
trip makers consider which park is most attractive to accomplish their
recreation activity. The Institute of Transportation Engineers (ITE) Trip
Generation Manual \citeyearpar{ite2019} contains trip attraction rates for public parks
that use as attraction terms the park acreage, number of picnic tables,
employees, and other variables. As with many land uses in Trip Generation, the
provided trip generation rates are based on a limited number of observational
samples and may not represent large-sample behavior \citep{Millard-Ball2015}.
Moreover, regression-based attraction rates isolated from trip production and
travel behavior ignore the geographical and behavioral contexts in which people
actually make trips to parks \citep{Barnard1987}: Though more people may come to
larger parks, a park cannot attract more people simply by becoming bigger.

There are limited examples of researchers using a destination choice model to
predict recreation attractions. \citet{Kinnell2006} apply a choice model to a survey
of park visitors in New Jersey, and estimate the relative attractiveness of park
attributes including playgrounds, picnic areas, and park acreage weighed against
the travel disutility and the relative crime rate at the destination. In a
similar study, \citet{Meyerhoff2010} model the urban swimming location choice for a
surveyed sample. In both studies, the researchers were attempting to ascertain
which attributes of a recreation generated the most positive utility, and
therefore which attributes should be prioritized for improvement. These studies
have not to our knowledge been previously referenced in discussions of park
accessibility.

\hypertarget{passive-location-data}{%
\subsection{Passive Location Data}\label{passive-location-data}}

The advent of large-scale mobile networks and the seemingly perpetual
association of unique devices with unique users has given researchers a new
opportunity to observe the movements and activity location patterns for large
subsets of the population \citep{Naboulsi2016}. Such passively collected movement
data --- sometimes referred to as ``Big Data'' --- is passively collected as a
by-product of other systems including cellular call data records \citep[e.g.,][]{Bolla2000, Calabrese2011}, probe GPS data \citep{Huang2015}, and more recently
Location Based Services (LBS) \citep{Roll2019, Komanduri2017}. LBS use a network of
mobile applications that obtain the users' physical location. A variety of
commercial vendors repackage, clean, and scale these data to population or
traffic targets and sell origin-destination matrices to researchers and
practitioners at relatively low prices. \citet{Monz2019}, for example, demonstrate
how passive device data can be used to accurately estimate trips to natural
recreation areas.

Passive origin-destination matrices are beginning to inform trip distribution
model development more directly as well. \citet{tf_idea} proposes one methodology,
where passive origin-destination matrices serve as a probabilistic sampling
frame for a simulated trip destination choice. \citet{Bernardin2018} employ a passive
origin-destination matrix as a shadow price reference in an activity-based
location choice model, iteratively adjusting the parameters of the choice
utilities to minimize the observed error between the matrix and the modeled
predictions. A similar method developed by \citet{Zhu2018} uses the passive dataset
directly, sampling 10,000 random trips from GPS traces of taxi trips in Shanghai
and estimating a destination choice model. Employing the passive data set in
this way provides the authors an opportunity to examine the choices of a
large sample of a small population (taxi passengers) as well as sufficient data
to estimate a ``constants-rich'' destination choice model with uniquely estimated
coefficients for each origin-destination pair. The Zhu and Ye methodology
suggests that a similar approach should apply in other contexts, including park
choice.

\hypertarget{methods}{%
\section{Methods}\label{methods}}

We describe our methods in this chapter.

\hypertarget{data}{%
\subsection{Data}\label{data}}

\hypertarget{applications}{%
\section{Applications}\label{applications}}

Some \emph{significant} applications are demonstrated in this chapter.

\hypertarget{example-one}{%
\subsection{Example one}\label{example-one}}

\hypertarget{example-two}{%
\subsection{Example two}\label{example-two}}

\hypertarget{final-words}{%
\section{Final Words}\label{final-words}}

We have finished a nice book.

\bibliography{book.bib}


\end{document}


