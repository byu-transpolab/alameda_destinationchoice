\documentclass{article}
	\def\papertitle{City parks and slow streets: a utility-based access and equity analysis}
	%\def\authors{Gregory S. Macfarlane, Christian Hunter, Austin Martinez, and Elizabeth Smith}
	\def\journal{Journal of Transport and Land Use}
	\def\paperid{8606}
% Define title defaults if not defined by user
\providecommand{\lettertitle}{Author Response to Reviews of}
\providecommand{\papertitle}{Title}
\providecommand{\authors}{Authors}
\providecommand{\journal}{Journal}
\providecommand{\paperid}{--}

\usepackage[includeheadfoot,top=20mm, bottom=20mm, footskip=2.5cm]{geometry}
\usepackage{booktabs}

% Typography
\usepackage[T1]{fontenc}
\usepackage{times}
%\usepackage{mathptmx} % math also in times font
\usepackage{amssymb,amsmath}
\usepackage{microtype}
\usepackage[utf8]{inputenc}
% Misc
\usepackage{graphicx}
\usepackage[hidelinks]{hyperref} %textopdfstring from pandoc
\usepackage{soul} % Highlight using \hl{}

\usepackage[round,authoryear]{natbib}

% Table

\usepackage{adjustbox} % center large tables across textwidth by surrounding tabular with \begin{adjustbox}{center}
\renewcommand{\arraystretch}{1.5} % enlarge spacing between rows
\usepackage{caption}
\captionsetup[table]{skip=10pt} % enlarge spacing between caption and table

% Section styles

\usepackage{titlesec}
\titleformat{\section}{\normalfont\large}{\makebox[0pt][r]{\bf \thesection.\hspace{4mm}}}{0em}{\bfseries}
\titleformat{\subsection}{\normalfont}{\makebox[0pt][r]{\bf \thesubsection.\hspace{4mm}}}{0em}{\bfseries}
\titlespacing{\subsection}{0em}{1em}{-0.3em} % left before after

% Paragraph styles

\setlength{\parskip}{0.6\baselineskip}%
\setlength{\parindent}{0pt}%

% Quotation styles

\usepackage{framed}
\let\oldquote=\quote
\let\endoldquote=\endquote
\renewenvironment{quote}{\begin{fquote}\advance\leftmargini -2.4em\begin{oldquote}}{\end{oldquote}\end{fquote}}

\usepackage{xcolor}
\newenvironment{fquote}
  {\def\FrameCommand{
	\fboxsep=0.6em % box to text padding
	\fcolorbox{black}{white}}%
	% the "2" can be changed to make the box smaller
    \MakeFramed {\advance\hsize-2\width \FrameRestore}
    \begin{minipage}{\linewidth}
  }
  {\end{minipage}\endMakeFramed}

% Table styles

\let\oldtabular=\tabular
\let\endoldtabular=\endtabular
\renewenvironment{tabular}[1]{\begin{adjustbox}{center}\begin{oldtabular}{#1}}{\end{oldtabular}\end{adjustbox}}


% Shortcuts

%% Let textbf be both, bold and italic
%\DeclareTextFontCommand{\textbf}{\bfseries\em}

%% Add RC and AR to the left of a paragraph
%\def\RC{\makebox[0pt][r]{\bf RC:\hspace{4mm}}}
%\def\AR{\makebox[0pt][r]{AR:\hspace{4mm}}}

%% Define that \RC and \AR should start and format the whole paragraph
\usepackage{suffix}
\long\def\RC#1\par{\makebox[0pt][r]{\bf RC:\hspace{4mm}}#1\par} %\RC
\WithSuffix\long\def\RC*#1\par{\textbf{\textit{#1}}\par} %\RC*
\long\def\AR#1\par{\makebox[0pt][r]{AR:\hspace{10pt}}\textit{#1}\par} %\AR
\WithSuffix\long\def\AR*#1\par{\textit{#1}\par} %\AR*


%%%
%DIF PREAMBLE EXTENSION ADDED BY LATEXDIFF
%DIF UNDERLINE PREAMBLE %DIF PREAMBLE
\RequirePackage[normalem]{ulem} %DIF PREAMBLE
\RequirePackage{color}\definecolor{RED}{rgb}{1,0,0}\definecolor{BLUE}{rgb}{0,0,1} %DIF PREAMBLE
\providecommand{\DIFadd}[1]{{\protect\color{blue}\uwave{#1}}} %DIF PREAMBLE
\providecommand{\DIFdel}[1]{{\protect\color{red}\sout{#1}}}                      %DIF PREAMBLE
%DIF SAFE PREAMBLE %DIF PREAMBLE
\providecommand{\DIFaddbegin}{} %DIF PREAMBLE
\providecommand{\DIFaddend}{} %DIF PREAMBLE
\providecommand{\DIFdelbegin}{} %DIF PREAMBLE
\providecommand{\DIFdelend}{} %DIF PREAMBLE
%DIF FLOATSAFE PREAMBLE %DIF PREAMBLE
\providecommand{\DIFaddFL}[1]{\DIFadd{#1}} %DIF PREAMBLE
\providecommand{\DIFdelFL}[1]{\DIFdel{#1}} %DIF PREAMBLE
\providecommand{\DIFaddbeginFL}{} %DIF PREAMBLE
\providecommand{\DIFaddendFL}{} %DIF PREAMBLE
\providecommand{\DIFdelbeginFL}{} %DIF PREAMBLE
\providecommand{\DIFdelendFL}{} %DIF PREAMBLE
%DIF END PREAMBLE EXTENSION ADDED BY LATEXDIFF

\begin{document}

% Make title
{\Large\bf \lettertitle}\\[1em]
{\huge \papertitle}\\[1em]
{\authors}\\
\textit{\journal}, \texttt{\paperid}\\
\hrule

% Legend
\hfill {\bfseries RC:} Reviewer Comment,\(\quad\) AR: \emph{Author Response}, \(\quad\square\) Manuscript text


We are grateful to the two anonymous reviewers for their continued review and consideration
of the manuscript.  
In this document we have highlighted \DIFadd{additions to the text of the manuscript with blue letters} and
\DIFdel{text removed from the manuscript with red letters.} These marks refer to 
changes resulting from this revision, and not since the initial manuscript submission.

A primary change from the previous revision is the new title of the
paper, which was changed in response to comments from Reviewer E discussed below. 
Some minor typographical errors were also corrected without discussion in this response.

\section{Reviewer A}

\RC The revised manuscript addresses most of my original concerns, and I believe it is close to being publishable. However, a few issues remain.

\AR We appreciate the time the reviewer has taken to help us improve the manuscript.

\subsection{Effects Interpretation}

\RC First, I agree with the authors’ interpretation of the segmented modeling results, i.e. that residents of block groups with high proportions of Black individuals are more sensitive to distance. However, I think the textual description (p. 17) of the marginal rate of substitution for distance and size from Table 3 (p. 18) might still be incorrect. Wouldn’t visitors be willing to travel 29\% farther to reach parks twice as big (0.391/1.358 from the Network Distance model)? See Yves Croissant’s (2010) “Estimation of Multinomial Logit Models in R: The mlogit Packages": \url{http://citeseerx.ist.psu.edu/viewdoc/download?doi=10.1.1.303.7401&rep=rep1&type=pdf}


\AR We believe the reviewer is correct. Consider the "Network Distance" model alternative utility equation,
\begin{equation}
V_{ij} = -1.358 \ln(\mathrm{miles}_{ij}) + 0.391 \ln(\mathrm{acres}_j)
\end{equation}
A unit analysis of this equation implies that the coefficient on distance is 
(for practical considerations) $1/\mathrm{\%\ miles}$ and the coefficient on 
size is $1/\mathrm{\%\ acres}$, so that $V_{ij}$ might be a unitless utility.
Thus, the rate at which individuals exchange area for distance would be
\begin{equation}
\frac{0.391\ 1/\mathrm{\%\ acres}}{-1.358 \ 1/\mathrm{\%\ miles}} = -0.288 \frac{\mathrm{\%\ miles}}{\mathrm{\%\ acres}}
\end{equation}
implying that a 29 percent increase in distance will be equivalent to a 100 percent
increase in park size. This has been corrected in the manuscript.

\begin{quote}
That is, individuals will travel further
distances to reach larger parks. The ratio of the estimated coefficients implies
that on average, people will travel \DIFdelbegin \DIFdel{twice as far }\DIFdelend \DIFaddbegin \DIFadd{0.288 times further }\DIFaddend to reach a park
\DIFdelbegin \DIFdel{3.47
times }\DIFdelend \DIFaddbegin \DIFadd{twice }\DIFaddend as large.

Table 3 also shows the results of the ``Park
Attributes'' model, which represents the presence of any sport field with a
single dummy variable, and the ``Sport Detail'' model, which disaggregates this
variable into facilities for different sports. The value of the size and
distance coefficients change modestly from the ``Network Distance'' model, with
the implied \DIFdelbegin \DIFdel{size to distance }\DIFdelend \DIFaddbegin \DIFadd{distance to size }\DIFaddend trade-off \DIFdelbegin \DIFdel{rising to 4.16}\DIFdelend \DIFaddbegin \DIFadd{changing to 0.243}\DIFaddend .
\end{quote}


\subsection{Additional Utility}


\RC Second, I appreciate the authors’ explanation for why there is some additional utility simply from having more options. However, the authors should explain in the text of the manuscript how they calculated that benefit (p. 21 in section 4.1). Specifically, it would be useful for less modelling-savvy readers to understand how the authors calculate the portion of the increased consumer surplus that comes just from having additional choices versus the portion that comes from the characteristics of the additional choices.

\AR It is not that there is a portion of utility that comes from more 
alternatives, and a portion that comes from characteristics of additional choices. 
Rather, more options always increase utility, even if those options are unrealistic.
It might be possible to restrict the scope of these benefits, for instance asserting
that closed streets more than $x$ miles from a home are not in the choice set, but
this would on some level undermine the purpose of logsum-based accessibility models
in the first place.
We have tried to explain this more strictly in the revised manuscript, 
including a citation to a common reference work.

\begin{quote}
One
property of logsum-based accessibility terms is that \DIFdelbegin \DIFdel{there is some benefit given
for simply having more options
}\DIFdelend \DIFaddbegin \DIFadd{additional options
will always increase the total logsum}\DIFaddend , whether or not those options are \DIFdelbegin \DIFdel{attractive in
any way}\DIFdelend \DIFaddbegin \DIFadd{realistic
\mbox{%DIFAUXCMD
\citep[p.~300]{benakiva}}\hskip0pt%DIFAUXCMD
}\DIFaddend . In this application, these benefits are \DIFaddbegin \DIFadd{apparently }\DIFaddend small, on the order
of 10 cents for most block groups away from where the street openings occurred.
\end{quote}

\subsection{Impacts Caveats}

\RC Third, the authors did a nice job caveating their estimated monetary benefits and explaining how they derived the cost coefficient (p. 21 in section 4.1). To make the monetary benefit estimation even more relatable to readers, they should also note how frequently the additional surplus accrues (one time? Annually? Every park trip?). I assume that residents would see this surplus each time they make a park trip?

\AR This is a good point, and one that we wrestled with in the study design. We have included an 
additional explanatory sentence,

\begin{quote}
It is
similarly not clear whether the benefits of improved park access should be
assigned at the person level, the household level, or the number of total park
trip makers in each block group\DIFaddbegin \DIFadd{. In theory, the benefits would accrue each time
a person chose to make a trip to a park or closed street, but to estimate this
would require a trip propensity or generation model, which we have not attempted
here}\DIFaddend . For consistency and simplicity, we assert that the benefit is assigned to
each household, and that persons receive a proportional share of the household
benefit.
\end{quote}

\RC Fourth, in discussing the distributional equities of the increased consumer surplus, the authors should note that in reality not all residents of each block group share the same utility equations (as indicated by the segmented models) and that the distributional equities might thus be different in reality.

\AR This is a good point. We have added the following sentence to the limitations
section:

\begin{quote}
 This interpretation could also explain some of the
non-intuitive response observed in our models, especially in regards to
playgrounds\DIFaddbegin \DIFadd{. Similarly, the use of individual-level estimation data would enable
heterogeneity in the benefits calculation: individuals could receive benefits for
the choice attributes they are observed to care most about, rather than an average
effect for all people as used in this research}\DIFaddend .
\end{quote}


\RC Fifth, the authors should better caveat the fact that street closures might not serve the same purposes as parks and thus might not be subject to the same utility functions. This is similar to the good point made by Reviewer E in the second sentence of comment 2.1.

\AR We have attempted to address this in response to comment 2.1 below.

\section{Reviewer E}


\RC I appreciate authors addressed some of my comments and comments. Some major issues still need to be addressed before the manuscript is accepted for publication. I would suggest revising and resubmitting it, with major revisions to be conducted.

\subsection{Surrounding Land Uses}

\RC On land uses at destinations (2.1): The authors stated in their response to my concern about not accounting for non-leisure or recreational trip purposes associated with subjects selecting a closed street as a destination the following: “It seems strange to consider including retail and restaurant opportunities surrounding a park in the destination choice model —if that is the suggestion — given that these facilities would have been shuttered during the COVID lockdowns when the street conversions were active.”

One excellent example of what I refer to is provided by our colleague Kelly Clifton et al. in their work titled ‘Development of destination choice models for pedestrian travel’ published a few years ago in Transportation Research Part A, and that include park availability and pedestrian environment measures, as well as other land-uses that may reduce the probability of a person walking to specific destinations. My comment does not suggest adding land uses surrounding parks, but accounting for potential destinations, people may reach the pedestrianized spaces in question. Of course, if there are no barriers to access these streets, and most, if not all, facilities such as restaurants, retail businesses, grocery stores, and the like, there is no question that such land uses are not needed to be included in the discrete choice model. Not being explicit about such nuances is a significant omission since not all cities that closed streets also closed all essential businesses – e.g., grocery shopping or pharmacies—and non-essential ones like restaurants and retail stores.

Therefore, the authors must provide empirical evidence supporting the claim that the type of facilities I highlighted in my comment on land uses were indeed closed in most if not all, places and times considered in the analysis. Such supporting evidence will ensure authors consider the potential estimation biases induced by omitting built-environment attributes that have been crucial in the large and growing body of literature on travel behavior and the built environment.


\AR Upon consideration of these valuable points and consulting the Clifton et al. reference,
we added surrounding land uses to our choice model in the form of a "Shops" variable.
This is simply a count of various amenity tags from OpenStreetMap as described below.

\begin{quote}
\DIFaddbegin \DIFadd{An additional amenity of parks that may affect their attractiveness the land use
that surrounds them \mbox{%DIFAUXCMD
\citep{clifton2016}}\hskip0pt%DIFAUXCMD
. Accordingly, we counted the number of features
in OpenStreetMap that are tagged with the amenities }\texttt{\DIFadd{bar}}\DIFadd{, }\texttt{\DIFadd{cafe}}\DIFadd{, }\texttt{\DIFadd{fast\_food}}\DIFadd{,
}\texttt{\DIFadd{restaurant}}\DIFadd{, }\texttt{\DIFadd{bank}}\DIFadd{, and }\texttt{\DIFadd{pharmacy}} \DIFadd{within 500 feet of the boundary of each park.
}

\DIFaddend
\end{quote}

\AR Overall, adding this variable to utility specification modestly improved the 
overall model fit, and the variable proves to be highly significant. We have added the 
comment describing the relevant model estimation results.

\begin{quote}
\DIFaddbegin \DIFadd{Both models also contain an estimate of the number of shops within 500 feet,
which is significant and positive. }\DIFaddend 
\end{quote}

\AR A further implication of adding this variable is that all the model estimates
and results calculates change slightly, but not enough to substantively alter the
central findings of the paper.

\subsection{Travel Mode of Access}

\RC On mode choice. The fact that authors required a measure of distance between home block group and park (or closed street) and chose to use the length of the walking path does not necessarily mean people in the data set provided by Streetlights include only people walking. Or does it? If it does, please clarify that; however, if it does not, please explain how the analysis inferred mode, or if any of the latter, then the research is seriously flawed. Again, my comment has nothing to do with estimated distances – Euclidian versus walking network distances.

\AR This research design is explicitly non-modal. The data provided by Streetlight Data
for this analyshis included all trips, not segmented by mode. 
We strongly disagree, however, that this decision must inevitably result in a flawed analysis. \\
\\
In some passive data sets, mode is inferred by the data provider. Consider that
we observe individuals in a neighborhood choose between two parks, $A$ and $B$. The 
passive data can tell us that of the 15 who chose park $A$, 5 drove and 10 walked,
and that of the 30 who chose park $B$, 15 drove and 15 walked. This data, however,
does not say anything about how those 45 individuals might have chosen to reach 
a new park $C$. It also does not reveal any information on what mode people who
chose park $A$ would have used if they had chosen park $B$.\\
\\
In our research, we present empirical evidence that people choose to visit 
parks that are closer to their home, regardless of the mode used to reach those 
parks. \\
\\
Do people choose modes first, and then destinations that can be reached by those 
modes? Or do they choose destinations first, and then choose the best to get there?
At some level this is a chicken-and-egg proposition, but in modern travel models, 
the issue is somewhat obviated by using the mode choice logsum as the travel impedance, 
replacing either a distance measure or, more commonly, a single-mode impedance. 
This allows people who are likely to walk to choose parks within walking distance, and
vice versa. But implementing this requires a mode choice model in addition to the
destination choice presented here, which we regard as beyond the question of this research.\\
\\
It is certainly not a valueless question. In other research currently under 
review, we have done precisely this. Table \ref{tab:park-models} presents estimates 
obtained in the same manner as the present research, but for a different metropolitan area.
The findings suggest that the mode choice logsum is marginally more predictive
than a single mode distance, as represented by the increased log-likelihood value
(which is significant in a Horowitz non-nested likelihood ratio test). It is 
possible that in a highly multi-modal scenario such as parts of Alameda county, 
the improvement would be even more substantial. But we believe the central findings
of the present study would remain inherently unchanged.

\begin{table}

\caption{\label{tab:park-models}Park Destination Choice Utilities - Other Research}
\centering
\begin{tabular}{lccccc}
\toprule
  & Car & MCLS & Attributes & All - Car & All - Logsum\\
\midrule
log(Network Distance) & -0.215(-95.949)** &  &  & -0.209(-69.212)** & \\
Mode Choice Logsum &  & 7.678(95.958)** &  &  & 7.450(69.216)**\\
log(Acres) &  &  & 1.308(77.120)** & 1.300(46.869)** & 1.301(46.858)**\\
Playground &  &  & 4.567(33.939)** & 4.476(30.127)** & 4.477(30.118)**\\
Volleyball &  &  & -0.369(-9.580)** & -0.663(-11.065)** & -0.664(-11.067)**\\
Basketball &  &  & -0.669(-15.625)** & -0.534(-7.632)** & -0.535(-7.642)**\\
Tennis &  &  & -0.549(-13.065)** & -0.884(-14.678)** & -0.886(-14.693)**\\
\midrule
Num.Obs. & 8,984 & 8,984 & 8,984 & 8,984 & 8,984\\
Log Likelihood & -9288.8 & -9284.7 & -11822.1 & -4774.9 & -4772.2\\
McFadden Rho-Sq & 0.569 & 0.569 & 0.451 & 0.778 & 0.778\\
\bottomrule
\multicolumn{6}{l}{\rule{0pt}{1em}t-statistics in parentheses. * p $<$ 0.5, ** p $<$ 0.1}\\
\end{tabular}
\end{table}



\subsection{Causal Effects}

\RC Let me clarify my comment on causality (2.3), which is more in the spirit of reflecting on language in light of research design limitations rather than the statistical method employed. I agree that the best parametrization to be used in a destination choice research context, as the authors well-explain in their methods section, is to use any of the available discrete choice models found in canonical work by prominent scholars such as Walker, McFadden, Train, and Ben-Akiva. However, a different question is whether interpreting a data set as cross-sectional to understand the potential impact of a policy will approximate well to an econometric analysis that uses data gathered in a natural experiment research design context or any other quasi-experimental research design. See, for instance, Doucette et al. 2021 ‘Initial impact of COVID-19’s stay-at-home order …’ in which the authors capitalized on data from Streetlight to address a different research question. The use of cross-sectional data rarely provides conclusive evidence that leads to cause-and-effect claims. Of course, what I say so far should not preclude researchers relying on cross-sectional analysis to publish their work – however, it must always include some cautionary notes, including:

The authors must acknowledge the limitations the data impose on the research and fine-tune the manuscript language avoiding any causal claims, starting by title.

The authors must reflect on their research design choice, and provide a recommendation for other researchers interested in harnessing the potential of Streetlight data, or the like.


\AR We are grateful for the clarifying statements from the reviewer. We agree 
that the study was not attempting to show a causal relationship, and it is possible
that title was confusing in this regard. Upon reflection, we suggest a new title
for the manuscript as:

\begin{quote}
\DIFdelbegin \DIFdel{If you build it who will come? Equity analysis of park system changes during COVID-19 using passive origin-destination data}\DIFdelend \DIFaddbegin \DIFadd{City parks and slow streets: a utility-based access and equity analysis} \DIFaddend 
\end{quote}

\AR We have also added an additional caveat to the limitations section,

\begin{quote}
\DIFaddbegin \DIFadd{Importantly, this research does not attempt to determine how many people
traveled to the converted streets, or how COVID-19 affected park destination
choice, though other researchers have used passive LBS data to answer similar
questions \mbox{%DIFAUXCMD
\citep{doucette}}\hskip0pt%DIFAUXCMD
.
}\DIFaddend 
\end{quote}

\subsection{Contribution}

\RC The authors must clearly state the contribution of the paper to scholarship, situating the work into current academic debates, besides the one I highlight next.

\AR We have added a clarifying statement to the conclusions, on what we consider
a primary contribution of this research.

\begin{quote}
In estimating these benefits, we applied an emerging technique to estimate park
choice preferences and utility from passive mobile device data. This technique
allowed a more nuanced measure of access that allowed us to consider the
converted streets as providing quantitatively different amenities than other city
parks \DIFaddbegin \DIFadd{and its application here is a primary contribution of this work}\DIFaddend .
\end{quote}


\subsection{Alternative Accessibility Measures}
\RC The paper should better differentiate the value of using alternative accessibility measures besides simplicity. One of the ongoing debates that cut-across transportation engineering, planning, and geography are how specific measures are more useful in planning or policy evaluation scenarios when observed behavior is not deemed adequate or valuable. This is particularly true when questions about distributive justice are paramount, and observed behavior only works to explain some inequalities but falls short in measuring un-meet travel demand needs. This is in addition to the ongoing concern by some scholars regarding the adoption of the concept in practice – see the most recent work on accessibility published in JAPA by Siddiq and Taylor, the work of Boisjoly and El-Geneidy on the same topic, and the work by A Paez titled ‘Measuring accessibility: positive and normative implementations of various accessibility indicators’ published in Transport Geography to start.

\AR We have updated the literature review to discuss the referenced work comparing accessibility metrics in terms of both practicality/simplicity and suitability to equity evaluation. 
The full details of these edits are perhaps too substantial to include in this response, but
the bulk of the change is in this paragraph

\begin{quote}
A major advantage of gravity-based accessibility measures lies in their
consistency with travel behavior theory: Gravity-based measures have their roots
in the trip distribution step of the traditional four-step travel demand
forecasting method, where trips originating in a particular zone are distributed
among destination zones, proportionate to each zone's gravity-based
accessibility. \DIFdelbegin \DIFdel{Urban scholars have used }\DIFdelend \DIFaddbegin \DIFadd{In spite of their theoretical advantages over cumulative-opportunity
measures, the practical advantages of }\DIFaddend gravity-based \DIFdelbegin \DIFdel{measures to explore the
spatial distribution of park access across Tainan City, Taiwan
\mbox{%DIFAUXCMD
\citep{chang2011exploring} }\hskip0pt%DIFAUXCMD
and to estimate the relationship between park access and
housing prices in Shenzhen, China \mbox{%DIFAUXCMD
\citep{wu2017spatial}}\hskip0pt%DIFAUXCMD
.
}\DIFdelend \DIFaddbegin \DIFadd{measures are less clear. Based on
their finding that cumulative-opportunity measures and gravity-based measures area
highly correlated, \mbox{%DIFAUXCMD
\citet{boisjoly2016daily} }\hskip0pt%DIFAUXCMD
argue that the greater simplicity of
cumulative-opportunity measures makes them appropriate for transportation planning
applications.
}\DIFaddend 

\DIFdelbegin \DIFdel{Some scholars have used }\DIFdelend \DIFaddbegin \DIFadd{\mbox{%DIFAUXCMD
\citet{paez2012measuring} }\hskip0pt%DIFAUXCMD
distinguish between positive and normative accessibility indicators,
where positive indicators are based on information about the degree to which
destinations are observed to be accessible and normative indicators also incorporate a
normative judgment of the degree to which destinations }\emph{\DIFadd{ought to be}} \DIFadd{accessible. They
develop a normative }\DIFaddend location-based  \DIFaddbegin \DIFadd{accessibility measure incorporating average trip
lengths into a cumulative opportunities measure, varying the isochrone threshold by
socioeconomic characteristics}\DIFaddend .
\end{quote}

\subsection{Position Accuracy}
\RC Authors must explain how position accuracy issues inherent from cell phone data, an issue often cited in the academic literature, could affect the analysis and conclusions, as well as the potential estimation biases induced by omitting built-environment attributes found not only at destinations that scholarship strongly suggests affecting the probabilities walking—for instance, sidewalks, land-uses along the road, etc.

\AR These are both very important points. We have added the following sentences 
to the limitations section:

\begin{quote}
\DIFaddbegin \DIFadd{There is even 
substantial question as to whether the device-holder was in the park
at all, given spatial error in the underlying LBS data and the potential
for conflating nearby activity locations.} \DIFaddend

$\cdots$

\DIFaddbegin \DIFadd{ Using the mode choice logsum  would also 
enable the inclusion of
qualitative path attributes (sidewalk quality, greenery, etc.) known to 
influence experienced travel impedance for particular modes of travel. \citep{clifton2016} \DIFaddend
\end{quote}

\subsection{Ben-Akiva Reference}
\RC Please cite the canonical work of Ben-Akiva in which was mentioned, probably
the first time, the use of utility-based models as a construct for the concept
of accessibility, including page number.

\AR We believe that at the latest, the idea of using changes in logsums to represent
user benefit calculations accredits to \citet{Williams1977a}. That said, we have 
included the Ben-Akiva and Lerman reference work on accessibility and choice set
monotonicity in two places, once in the methodology and again in response to 
issue 1.2 above.

\begin{quote}
 Note also that the logsum increases
with the size of the choice set: if a new alternative \(q\) is added to \(J\), then
\(\ln\sum_{j\in J}\exp(V_j) < \ln(\sum_{j\in J}\exp(V_j) + \exp(V_q))\) for
any value of \(V_q\) \DIFaddbegin \DIFadd{\mbox{%DIFAUXCMD
\citep[p.~300]{benakiva}}\hskip0pt%DIFAUXCMD
}\DIFaddend .
\end{quote}

\bibliography{book}
\bibliographystyle{abbrvnat}
\end{document}
